\documentclass[notheorems,serif,table,compress]{beamer}  %dvipdfm选项是关键,否则编译统统通不过
%%------------------------常用宏包------------------------
%%注意, beamer 会默认使用下列宏包: amsthm, graphicx, hyperref, color, xcolor, 等等
\usepackage{fontspec,xunicode,xltxtra}  % for XeTeX
\usepackage{verbatim}
\usepackage{mathabx}
\usepackage{latexsym}
\usepackage{amsfonts,amssymb}
\usepackage{styles/iplouclistings}
\usepackage{fancybox}
\usepackage{colortbl}
\usepackage{tcolorbox}
%\usepackage[T1]{fontenc}
%\usepackage{bookman}
\usepackage{subfigure}
\usepackage{hyperref}
\usepackage{listings}
\usepackage{animate}
\usepackage[absolute,overlay]{textpos}
\usepackage{graphicx}
\usepackage{tikz}
\usepackage[americaninductors,europeanresistors]{circuitikz}
\usepackage{tikz}
\usepackage{fancybox}     %% 定义zhushadow时用到
\usepackage{pifont} %ding用到
\newsavebox{\mysaveboxOne}  %%为了在only中使用lstlisting
\newsavebox{\mysaveboxTwo}
\newsavebox{\mysaveboxThree}
\newsavebox{\mysaveboxFour}
\newsavebox{\mysaveboxFive}
\newsavebox{\mysaveboxSix}
\newsavebox{\mysaveboxSeven}
\newcommand\zhushadow[2][purple]{\hskip5pt\shadowbox{\color{#1}\small\kai #2\vspace{3mm}}}

%%------------------------ThemeColorFont------------------------
%% Presentation Themes
% \usetheme[<options>]{<name list>}
%\usetheme{Madrid}
\usetheme{Berkeley}
%% Inner Themes双精度计算
% \useinnertheme[<options>]{<name>}
%% Outer Themes
% \useoutertheme[<options>]{<name>}
%\useoutertheme{miniframes} 
%% Color Themes 
%\usecolortheme[<options>]{<name list>}
%% Font Themes
\usefonttheme{serif}
\setbeamertemplate{background canvas}[vertical shading][bottom=white,top=structure.fg!7] %%背景色, 上25%的蓝, 过渡到下白.
\setbeamertemplate{theorems}[numbered]
\setbeamertemplate{navigation symbols}{}   %% 去掉页面下方默认的导航条.
\usepackage{styles/zhfontcfg}
%\setsansfont[Mapping=tex-text]{文泉驿正黑}  %% 需要fontspec宏包
     %如果装了Adobe Acrobat,可在font.conf中配置Adobe字体的路径以使用其中文字体
     %也可直接使用系统中的中文字体如SimSun,SimHei,微软雅黑 等
     %原来beamer用的字体是sans family;注意Mapping的大小写,不能写错
     %设置字体时也可以直接用字体名,以下三种方式等同:
     %\setromanfont[BoldFont={黑体}]{宋体}
     %\setromanfont[BoldFont={SimHei}]{SimSun}
     %\setromanfont[BoldFont={"[simhei.ttf]"}]{"[simsun.ttc]"}
%%------------------------MISC------------------------
\graphicspath{{figures/}}         %% 图片路径. 本文的图片都放在这个文件夹里了.
%%------------------------listing------------------------
%\lstset{language=[LaTeX]TeX,Python}
%%------------------------正文------------------------
\begin{document}
\XeTeXlinebreaklocale "zh"         % 表示用中文的断行
\XeTeXlinebreakskip = 0pt plus 1pt % 多一点调整的空间
%%----------------------------------------------------------
%% This is only inserted into the PDF information catalog. Can be left
%% out.
%%%
%% Delete this, if you do not want the table of contents to pop up at
%% the beginning of each subsection:
%\AtBeginSection[]{                              % 在每个Section前都会加入的Frame
%  \frame<handout:0>{
%    \frametitle{Contents}\small
%    \tableofcontents[current,currentsubsection]
%  }
%}
%
%\AtBeginSubsection[]                            % 在每个子段落之前
%{
%  \frame<handout:0>                             % handout:0 表示只在手稿中出现
%  {
%    \frametitle{Contents}\small
%    \tableofcontents[current,currentsubsection] % 显示在目录中加亮的当前章节
%  }
%}

\setbeamertemplate{caption}{\raggedright\insertcaption\par}

%%----------------------------------------------------------
\logo{\includegraphics[scale=0.13]{ouclogo.png}}
\title{Histogram Processing}
%\subtitle{Bottom-Up Saliency Detection Model Based on Human Visual Sensitivity and Amplitude Spectrum}
\subtitle{直方图处理}
\author[]{\textcolor{olive}{TangNing}}
\institute[CVBIOUC]
{
\small\textcolor{violet}{CVBIOUC\\
%Ocean University of China\\
\url{http://vision.ouc.edu.cn/~zhenghaiyong}}
}
%\date[]{}
%\titlegraphic{
%\includegraphics[height=1.0cm]{ouc-logo.jpg}}
\frame{ \titlepage }
%%----------------------------------------------------------
%\section*{Contents}
\frame{\frametitle{Contents} \tableofcontents}
%%----------------------------------------------------------
\def\hilite<#1>{\temporal<#1>{\color{blue!15}}{\color{black}}{\color{black}}}
\newcommand{\shadow}[2][purple]{\hskip5pt\shadowbox{\color{#1}\small \kai #2\vspace{3mm}}}
\newcommand{\colorrbox}[2][purple]{\doublebox{\color{#1}\small \kai#2}}

%============================================================================

\section{Introduction}

%==========================================================================

\begin{frame}[fragile]
\frametitle{Histogram}
\begin{enumerate}
\item {\color{blue}The definition of histogram:}\\
\begin{center}
$h(r_k) = n_k$ \quad $k=0,1,...,L-1$
\end{center}
where $r_k$ is the kth intensity value and $n_k$ is the number of pixels in the image with intensity $r_k$.
\item {\color{blue}Normalize a hitstogram:}\\
\begin{center}
$p(r_k)=\cfrac{n_k}{MN}$\quad $k=0,1,...,L-1$
\end{center}
$p(r_k)$ is the probability of occurrrnce of intensity level $r_k$ in an image.The sum of all components of a normalized histogram is equal to 1.
\end{enumerate}

\end{frame}
\begin{frame}
\frametitle{Four basic image types and their correponding histograms}
\begin{figure}
\begin{center}
\subfigure{
\includegraphics[width=0.2\textwidth]{1.png}}
\subfigure{
\includegraphics[width=0.25\textwidth]{11.png}}
\end{center}
\caption{dark image}
\begin{center}
\subfigure{
\includegraphics[width=0.2\textwidth]{2.png}}
\subfigure{
\includegraphics[width=0.25\textwidth]{22.png}}
\end{center}
\caption{light image}
\end{figure}
\end{frame}
\begin{frame}
\frametitle{Four basic image types and their correponding histograms}
\begin{figure}
\begin{center}
\subfigure{
\includegraphics[width=0.2\textwidth]{3.png}}
\subfigure{
\includegraphics[width=0.25\textwidth]{33.png}}
\end{center}
\caption{low contrast image}
\begin{center}
\subfigure{
\includegraphics[width=0.2\textwidth]{4.png}}
\subfigure{
\includegraphics[width=0.25\textwidth]{44.png}}
\end{center}
\caption{high contrast image}
\end{figure}
\end{frame}
%============================================================================
\section{Histogram Equalization}
%===============================================================================
\begin{frame}
\frametitle{Transformations}
Transformations(intensity mappings)of the form:\\
\begin{equation}
s=T(r)\quad 0\leq r\leq L-1
\end{equation}\\
{\color{blue}(a)}$T(r)$is a monotonically increasing function in the interval $0\leq r\leq L-1$ .\\
{\color{blue}(b)}$0\leq T(r)\leq L-1$ for $0\leq r\leq L-1$ .
\end{frame}
\begin{frame}
\frametitle{Histogram Equalization}
\begin{align} \label{eq:eps}
&s_k=T(r_k)=(L-1)\sum_{j=0}^k p_r(r_j)  \\
&=\cfrac{(L-1)}{MN}\sum_{j=0}^k n_j \quad k=0,1,2...,L-1\nonumber
\end{align}
a processed image is obtained by mapping rach pixel in the input image with intensity    $r_k$ into a corresponding pixel with level $s_k$ in the output image,using(\ref{eq:eps}).The transformation $T(r_k)$ in this equation is called a \emph{histogram equalization}.
\end{frame}
\begin{frame}
\frametitle{Result}
\begin{figure}
\begin{center}
\subfigure[]{
\includegraphics[width=0.12\textwidth]{origin.jpg}}
\subfigure[]{
\includegraphics[width=0.25\textwidth]{origin.png}}
\end{center}
\caption{(a)Origin image.(b)Histogram of (a).}
\begin{center}
\subfigure[]{
\includegraphics[width=0.12\textwidth]{histeq.jpg}}
\subfigure[]{
\includegraphics[width=0.25\textwidth]{histeq.png}}
\end{center}
\caption{(c)Histogram-equalized image.(d)Histogram of (c).}
\end{figure}
\end{frame}
%============================================================================
\section{Histogram Matching}
%===============================================================================
\begin{frame}
\frametitle{Histogram Matching}
{\color{blue}(a)}Obtain the values of $s$ by using the histogram equalization transformation:
\begin{align} 
&s_k=T(r_k)=(L-1)\sum_{j=0}^k p_r(r_j)  \\
&=\cfrac{(L-1)}{MN}\sum_{j=0}^k n_j \quad k=0,1,2...,L-1\nonumber
\end{align}
{\color{blue}(b)}Compute all values of the transformation function $G$ using the specified PDF:
\begin{equation} 
G(z_q)=(L-1)\sum_{i=0}^q p_z(z_i) 
\end{equation}
\end{frame}
\begin{frame}
\frametitle{Histogram Matching}
{\color{blue}(c)}Find the corresponding value of$z_q$ so that $G(z_q)$ is closest to $s_k$ :
\begin{equation} 
G(z_q)=s_k
\end{equation}
{\color{blue}(d)}Get the desired value $z_q$ by obtaining the inverse transformation:
\begin{equation} 
z_q=G^{-1}(s_k)
\end{equation}
\end{frame}
\begin{frame}
\frametitle{Result}
\begin{figure}
\begin{center}
\subfigure[]{
\includegraphics[width=0.12\textwidth]{origin.jpg}}
\subfigure[]{
\includegraphics[width=0.25\textwidth]{origin.png}}
\end{center}
\caption{(a)Origin image.(b)Histogram of (a).}
\begin{center}
\subfigure[]{
\includegraphics[width=0.12\textwidth]{histsp.jpg}}
\subfigure[]{
\includegraphics[width=0.25\textwidth]{histsp.png}}
\end{center}
\caption{(c)Histogram-specified image.(d)Histogram of (c).}
\end{figure}
\end{frame}
%============================================================================
\section{Local Histogram Equalization}
%===============================================================================
\begin{frame}
\frametitle{Local Histogram Equalization}
Use local histogram equalization with a neighborhood of size $3\times3$:
\begin{figure}
\begin{center}
\subfigure[]{
\includegraphics[width=0.2\textwidth]{input.png}}
\subfigure[]{
\includegraphics[width=0.2\textwidth]{output.jpg}}
\subfigure[]{
\includegraphics[width=0.2\textwidth]{book.png}}
\end{center}
\caption{(a)Origin image.(b)Result of local histogram equalization applied to (a).}
\end{figure}
\end{frame}

%============================================================================
\section{Using Histogram Statistics for Image Enhancement}
%===============================================================================
\begin{frame}
\frametitle{Using Histogram Statistics for Image Enhancement}
Use local histogram statistics with a neighborhood of size $3\times3$:
\begin{figure}
\begin{center}
\subfigure[]{
\includegraphics[width=0.2\textwidth]{imgehorigin.png}}
\subfigure[]{
\includegraphics[width=0.2\textwidth]{imgeh.jpg}}
\subfigure[]{
\includegraphics[width=0.2\textwidth]{wrong.jpg}}
\end{center}
\caption{(a)Origin image.(b)Image enhanced using local histogram statistics.}
\end{figure}
\end{frame}


\end{document}
